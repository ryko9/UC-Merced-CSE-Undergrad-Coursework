David Wang
CSE 15

Page 78 
1)
	If p then q

	p -> q 

	if p is true, therefore q is true.

	statement q, 'Socrates is mortal'  is true because statement p, 'Socrates is human' is true in this case.

2)
	If not p then not q

	q is true, therefore p is also true.

	q = George is a spider.
	q = George has eight legs.

	Since george is a spider is true, george has eight legs should also be true.

5)


	p = Randy works hard
	q = Randy is a dull boy.
	r = Randy will not get the job.


	p --> q
	q --> not r
	therefore not r

	Since p is true, q is true, and therefore not r is also true.


6)

	p = it rained
	q = it is foggy
	r = the sailing race will be held.
	s = trophy rewarded

	not p \/ not q ---> r 
	r ---> s

	not s
	therefore p

	not p or not q implies r, and r implies s. 

	Since not s happened, r didn't happen, which means that it rained since not p or not q implies r.

Page 91

1) 
	x + y = z
	1 + 3 = 4
	3 + 5 = 8

	Thus the sum of two odd integers is ALWAYS even 

2)
	2 + 4 = 6
	4 + 4 = 8

	Thus, the sum of two even integers is ALWAYS even

3)
	Assume n is even

		Thus, n = 2k, for some k
		n^2 
		= ( 2k )^2 
		= 4k^2 
		= 2( 2k^2 )

	Since  n^2 is 2 times an integer, it is even
6)
	x * y = z 
	Assume x and y are odd.
	Z will will always be odd since both x and y are odd. For ex/ 7 * 9 = 63   9 * 9 = 81 

9)
	Let  q be the irrational number and r be the rational number. Assume their sum is rational such in  q+ r = s
	where s is a rational number. Then q must equal s - r. But, the sum of two rational numbers must be rational 
	and we have an irrational number on the left. This is a contradictional therefore  q + r must be irrational	
	
11)
	Let x = sqrt(2) and y = sqrt(2)
	x * y = z where z is an irrational number
	But,
		sqrt(2) * sqrt(2) = 2 where z is a rational number

	So it is disproved

	
17)
	a) If n is odd, then n^3 + 5 is even. Assume that n is odd. Then we can write n = 2k + 1 for some integer k. 

		 n^3 + 5 = (2k + 1)^3 + 5 
			     = 8k^3 + 12k^2 + 6k + 6 
			     = 2(4k^3 + 6k^2 + 3k + 3).

	  n^3 + 5 is 2x some integer, so it is EVEN
	
	b) 
	
	Suppose that n^ 3 + 5 is odd and that n is odd. Since n is odd, and the product of odd numbers is odd, in two steps we see that n^3
	is odd. When we  subtract we conclude that 5 which is the difference of the two odd numbers n^3 + 5 and n is even; This is false,
	therefore the proof of contradiction is true.


18) 
	a)  

	Assume n is odd and show that 3n + 2 is odd.
	n = 2k +1 for some integer k.
	3n +2 = 3(2k+1) + 2
		 = 6k +5
		 = 6k +4 +1
		 =2(3k+2) + 1

	 3n+2 is 2 times an integer + 1, so it is odd. 

	b) The statement is false when 3n + 2 is even, and n is odd, so assume 3n + 2 is even, and n is odd.
	n = 2k +1 for some integer k.
	3n +2 = 3(2k+1) + 2
		 = 6k +3 +2
		 = 6k +4 +1
		 = 2(3k+2) + 1 

	 3n+2 is odd, even though it was assumed to be even originally.
		
